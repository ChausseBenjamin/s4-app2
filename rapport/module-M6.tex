\section{Module M6}

\note{votre lecteur ne connait pas votre démarche ni
pourquoi un tel résultat démontre le bon fonctionnement – vous devez l’expliquez brièvement.}

\note{Nous ne consulterons pas le dépôt de vos fichiers Xilinx (sauf
dans des cas particuliers) ce qui signifie que votre rapport doit être complet en lui-même.}

\quickfigure{Machine à états finis du module M6}{mef-m6}{}{assets/m6-mef-mermaid.pdf}

\todo{Schéma-bloc du système}

\todo{Description sommaire du fonctionnement}

\todo{Plan de vérification}

\todo{Validation via une simulation avec le banc de test fourni. Vous devez démontrer le
fonctionnement de vos modules au travers de simulations et expliquez en quoi vous
répondez aux spécifications. Faites bon usage des curseurs, formattage des données et
agrandissements pour appuyer vos propos.}
