\section{Module M6}

\note{votre lecteur ne connait pas votre démarche ni
pourquoi un tel résultat démontre le bon fonctionnement – vous devez l'expliquez brièvement.}

\note{Nous ne consulterons pas le dépôt de vos fichiers Xilinx (sauf
dans des cas particuliers) ce qui signifie que votre rapport doit être complet en lui-même.}

\quickfigure{Machine à états finis du module M6}{mef-m6}{}{assets/m6-mef-mermaid.pdf}

\todo{Schéma-bloc du système}

\subsection{Description du fonctionnement}

Le module M6 effectue une estimation de la puissance du signal audio en
appliquant une moyenne glissante exponentielle sur les échantillons entrants.
Le calcul est déclenché à chaque front montant de l'horloge \verb|i_bclk|, si
le signal d'activation \verb|i_en| est à 1.\\

À chaque activation, la puissance instantanée est obtenue en élevant au carré
l'échantillon courant \verb|i_ech|. Cette valeur est ensuite combinée avec la
puissance historique pondérée, laquelle est approximée par un facteur de
31/32 via un ensemble de décalages binaires successifs (équivalent à un
filtrage passe-bas exponentiel).\\

Le résultat de cette moyenne glissante est accumulé dans le registre
\verb|oldest_power|, et la sortie \verb|o_param| correspond à une portion des
bits de poids fort de cette somme, servant d'estimation compacte de la
puissance audio pour affichage ou traitement subséquent.\\

% \todo{Description sommaire du fonctionnement}
\todo{Plan de vérification}

\todo{Validation via une simulation avec le banc de test fourni. Vous devez démontrer le
fonctionnement de vos modules au travers de simulations et expliquez en quoi vous
répondez aux spécifications. Faites bon usage des curseurs, formattage des données et
agrandissements pour appuyer vos propos.}
