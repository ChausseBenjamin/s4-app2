\documentclass[a11paper]{article}
% xelatex

\usepackage{subcaption}
\usepackage{tabularx}
\usepackage{titlepage}
\usepackage{document}
\usepackage{booktabs}
\usepackage{multicol}
\usepackage{multirow}
\usepackage{float}
\usepackage{varwidth}
\usepackage{graphicx}
\usepackage{siunitx}
\usepackage{pifont}
\usepackage{subcaption}
% \usepackage[toc,page]{appendix}
\usepackage[usenames,dvipsnames]{xcolor}

\title{Rapport d'APP}

\class{Logique Combinatoire}
\classnb{GEN420 \& GEN430}

\teacher{Berthié Gouin-Ferland \& Nikola Zelovic}

\author{
  \addtolength{\tabcolsep}{-0.4em}
  \begin{tabular}{rcl} % Ajouter des auteurs au besoin
      Benjamin Chausse & -- & CHAB1704 \\
      Shawn Couture    & -- & COUS1912 \\
  \end{tabular}
}

\newcommand{\todo}[1]{\begin{color}{Red}\textbf{TODO:} #1\end{color}}
\newcommand{\note}[1]{\begin{color}{Orange}\textbf{NOTE:} #1\end{color}}
\newcommand{\fixme}[1]{\begin{color}{Fuchsia}\textbf{FIXME:} #1\end{color}}
\newcommand{\question}[1]{\begin{color}{ForestGreen}\textbf{QUESTION:} #1\end{color}}

% Checkboxes
\setlength{\fboxsep}{1pt}
\newcommand{\cbox}{\fbox{\phantom{\ding{51}}}}
\newcommand{\cboxtick}{\fbox{\ding{51}}}%

\newcommand{\quicktable}[4]{
  \begin{table}[H]
    \footnotesize
    \centering
    \caption{#1}
    \label{tab:#2}
    \begin{tabular}{#3}
      #4
    \end{tabular}
  \end{table}
}

\newcommand{\quickfigure}[4]{
  \begin{figure}[H]
    \centering
		\includegraphics[width=#3\textwidth]{#4}
    \caption{#1}
    \label{fig:#2}
  \end{figure}
}

\begin{document}
\maketitle
\newpage
\tableofcontents
\newpage

\note{Un total de 4 pages est permis, excluant la page titre, la table des matières, l'introduction,
la conclusion et les références.}

\section{Introduction}

Dans le cadre de l'APP2 de la session S4, il est demandé de compléter le
développement d'une pédale d'effets de guitare numérique à l'aide d'un FPGA
sur la carte Zybo. Plus précisément, ce rapport documente la conception,
l'implémentation et la vérification de deux modules essentiels au système:
le décodeur I2S (M1), responsable de la désérialisation du signal audio reçu,
et le module de calcul de la puissance du signal (M6). Ces travaux ont été
réalisés en respectant les contraintes de synchronisation, de modularité et
de formalisme en VHDL, selon les normes enseignées. Le rapport présente les
diagrammes d'états, les schémas-blocs, les plans de vérification ainsi que
les résultats de simulation attestant de la conformité fonctionnelle des
modules développés.

\section{Module M1}

\note{votre lecteur ne connait pas votre démarche ni
pourquoi un tel résultat démontre le bon fonctionnement – vous devez l’expliquez brièvement.}

\note{Nous ne consulterons pas le dépôt de vos fichiers Xilinx (sauf
dans des cas particuliers) ce qui signifie que votre rapport doit être complet en lui-même.}

\todo{Description du fonctionnement du décodeur avec court texte approprié se basant sur
le schéma-bloc fourni (il n’est pas nécessaire de l’inclure dans le rapport)}

\todo{Diagramme d’états de la MEF}

\todo{Plan de vérification}

\todo{Validation via une simulation avec le banc de test fourni. Vous devez démontrer le
fonctionnement de vos modules au travers de simulations et expliquez en quoi vous
répondez aux spécifications. Faites bon usage des curseurs, formattage des données et
agrandissements pour appuyer vos propos.}

\section{Module M6}

\note{votre lecteur ne connait pas votre démarche ni
pourquoi un tel résultat démontre le bon fonctionnement – vous devez l'expliquez brièvement.}

\note{Nous ne consulterons pas le dépôt de vos fichiers Xilinx (sauf
dans des cas particuliers) ce qui signifie que votre rapport doit être complet en lui-même.}

\quicktable{Plan de vérification du module M6 -- Calcul de puissance}{mef-m6}{p{4cm}p{5cm}p{6cm}l}{
  \toprule
  \textbf{Objectif Ciblé} &
  & \multirow{2}{*}{\textbf{Valider la MEF calcul de puissance d'ondes}}
  & \\
  Condition à proscrire
  & Ondes ayant une fréquence plus haute que \SI{24}{\kilo\hertz}
  &
  & \\
  \midrule
  \textbf{Test}
  & \textbf{Action}
  & \textbf{Résultat attendus}
  & \cboxtick \\
  \midrule
  Condition initial
  & Mettre les interrupteurs à 0010 et une onde sans amplitude
  & La puissance en sortie reste à 0.
  & \cbox\\
  Reset initial
  & \texttt{BTN3} à 1 et onde sans amplitude.
  & La valeur de la puissance devient 0 et reste à 0 après.
  & \cbox\\
  Test d'une onde sinuosïdale
  & Mettre une onde sinus de \SI{1}{\kilo\hertz}. À amplitude maximum
  & La puissance augmente vers le maximum et reste stable.
  & \cbox\\
  Test d'une onde sinuosïdale avec bruit
  & Mettre une onde sinus de \SI{1}{\kilo\hertz} avec un peu de bruit. À amplitude maximum
  & La puissance augmente vers le maximum et reste plus ou moins stable.
  & \cbox\\
  Test d'une onde Carré
  & Mettre une onde carré de \SI{1}{\kilo\hertz} à amplitude maximum
  & La puissance augmente vers le maximum et reste stable. Observation de divots lors de changement de signe.
  & \cbox\\
  Test de haute fréquences
  & Mettre une onde sinus de \SI{20}{\kilo\hertz}
  & La puissance oscille en fonction de l'amplitude du signal.
  & \cbox\\
  Test de saturation
  & Un signal d'amplitude maximum envoyé en continue
  & La puissance atteint le maximum après quelques échantillons et aucun overflow se produit.
  & \cbox\\
  Test d'amplitude fix basse
  & Onde carrée d'amplitude 0.5 à -0.5 envoyé en entrée
  & la puissance atteint la moitié du maximum après quelques échantillons
  & \cbox\\
  \bottomrule
}

\begin{figure}[H]
  \centering
  \begin{subfigure}{.496\linewidth}
    \centering
    \includegraphics[width=\textwidth]{assets/chrono-m6-1000hz-sin-noisy.png}
    \caption{Signal sinusoïdal bruité à \SI{1}{\kilo\hertz}}
    \label{fig:chrono-m6-1000hz-sin-noisy}
  \end{subfigure}
  \begin{subfigure}{.496\linewidth}
    \centering
    \includegraphics[width=\textwidth]{assets/chrono-m6-reset-and-20000hz.png}
    \caption{Post-réinitialisation et signal à \SI{20}{\kilo\hertz}}
    \label{fig:chrono-m6-reset-and-20000hz}
  \end{subfigure}
  \begin{subfigure}{.496\linewidth}
    \centering
    \includegraphics[width=\textwidth]{assets/chrono-m6-saturation-works.png}
    \caption{Saturation du module}
    \label{fig:chrono-m6-saturation-works}
  \end{subfigure}
  \begin{subfigure}{.496\linewidth}
    \centering
    \includegraphics[width=\textwidth]{assets/chrono-m6-square-5000hz-2in1.png}
    \caption{Signal carré à 5000 Hz avec deux canaux}
    \label{fig:chrono-m6-square-5000hz-2in1}
  \end{subfigure}
  \caption{Chronographes du module M6 sous différentes conditions}
\end{figure}

% \quickfigure{Machine à états finis du module M6}{mef-m6}{}{assets/m6-mef-mermaid.pdf}

\todo{Schéma-bloc du système}

\subsection{Description du fonctionnement}

Le module M6 effectue une estimation de la puissance du signal audio en
appliquant un filtrage exponentiel à moyenne glissante sur les échantillons
entrants. Le calcul s'effectue à chaque front montant de l'horloge
\verb|i_bclk|, lorsque le signal \verb|i_en| est à 1. \\

À chaque activation, la puissance instantanée est calculée en élevant au
carré l'échantillon courant \verb|i_ech|. Cette nouvelle valeur est ensuite
ajoutée à l'historique de puissance, qui est préalablement réduite par un
facteur d'oubli de 31/32. Ce facteur assure que les résultats récents ont
plus de poids que les anciens, tout en limitant la croissance de la valeur
accumulée afin d'éviter un débordement (\emph{overflow}). \\

Le résultat final est stocké dans le registre \verb|oldest_power|. Pour
simplifier l'affichage ou l'utilisation de cette puissance, seuls les bits de
poids fort (\verb|oldest_power(46 downto 39)|) sont extraits et envoyés sur
la sortie \verb|o_param|. \\

\\

Le fonctionnement réponds plus ou moins aux spécifications. À cause de la méthode utilisé, la mesure de la puissance ne se stabilise pas assez lentement pour donner un nombre fixes à des basses fréquences (chronogramme A). Cependant ce chronogramme démontre la résistance au bruit. Le chronogramme B démontre le bon fonctionnement du reset (le tout reste à zéro jusqu’à la relâche du signal en rouge). Le chronogramme C démontre que le calcul monte tranquillement vers le maximum (prouvant l'utilisation d'accumulation) et n'éprouve aucun dépassement restant stable. Le chronogramme D démontre que la puissance reste stable et s'adapte aux changements d'amplitude du signal entrant dans le module. Le chronogramme B à une sinus à l'entrée, mais le graphique à été réalisé avec la fonction d'affichage « hold » se qui donne l'impression d'une onde carré. Les spécification sont donc atteinte car un facteur de 31/32 est utilisée et une accumulation en continue est présente. Cependant, l'angle d'approche serait à revoir afin d'avoir des lectures stables avec des ondes sinusoïdales.

% \todo{Description sommaire du fonctionnement}
\todo{Plan de vérification}

\todo{Validation via une simulation avec le banc de test fourni. Vous devez démontrer le
fonctionnement de vos modules au travers de simulations et expliquez en quoi vous
répondez aux spécifications. Faites bon usage des curseurs, formattage des données et
agrandissements pour appuyer vos propos.}


\section{Conclusion}
\todo{write the fucking thing}

\end{document}
